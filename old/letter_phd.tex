



Dear Dr. Edith Ngai and members of the selection committee:

My name is Mikael Dubik and I'm writing to you to apply for the position of PhD on data analytics and machine learning for smart cities in Uppsala University. I graduated from Uppsala University with a Masters degree in Computer Science under direction of Senior Lecturer Matteo Magnani, who oversees UU-Infolab. At Uppsala, the main courses I took were focused on programming backend systems and data mining / machine learning. I am 28 years old, currently living in Iceland, Reykjavík working for Marorka My main interest lies in data collection and analysis plus backend systems. I spend my free time working on machine learning problems, statistics, cooking and being outdoors (mostly cycling). I enjoy working in a team but I also have a proven record of independent work and research.

Ever since I started my studies in Computer Science I have been fascinated by automation and of thinking machines. I did a project for my bachelor thesis on image recognition and got experience working with OpenCV and Haar classifiers. After that I worked for a flight search engine were I got experience on data collecting, loading and manipulation. I also worked briefly for University of Iceland in integrating voice recognition software into a virtual environment. Currently I work at Marorka, which specializes itself in maritime analytics, emission reduction and ship fuel savings. At Marorka I got firsthand experience working in a cloud computing environment (AWS Systems and services) were I designed an ETL infrastructure to accept and analyze data, with Apache Kafka.

My main research contribution to data analytics and machine learning is in my dissertation, ``A comparative evaluation of state-of-the-art community detection algorithms for multiplex networks" (diva2:1154983) where network clustering algorithms are implemented (a type of unsupervised learning) to analyze synthetic and real life networks. We analyzed different types of networks, synthetic and real, and since then I've been fascinated by these algorithms. I believe these algorithms have potential to be more developed and popularized so they can be used in more applications, for example in smart city planning. This dissertation gave me the necessary experience on reading academic papers and extracting the important information as references. 

My future research plans involve development and implementation of supervised and unsupervised (possibly a combination of both) machine learning algorithms and developing my understanding of them. I wish to specialize in this field so I can become an expert in machine learning. With my professional programming background, high interest and experience in data analysis and machine learning I believe I have the necessary qualification to contribute new information and software to this project.

Attached to this letter is my CV that explains in detail my past achievements and experience. In addition, my reviewer has written a Letter of Recommendation which describes his experience working with me. I am available for further talks in a personal interview if more information is requested from me. I would like to thank you for your time reading this letter and I hope to hear from you again.

Sincerely,




Mikael Dubik  
mkdubik@gmail.com, +354 8484903
